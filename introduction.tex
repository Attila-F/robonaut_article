\section{Introduction}
\label{sec:Introduction}

% What is the problem?

The RobonAUT is an annual contest of the Department of Automation and Applied Informatics of the Budapest University of Technology and Economics. Each year many teams design and build autonomous model cars to battle on an obstacle course, and compete each other on a race track. The robots must be completely autonomous, lack any remote control and any sort of external intervention during the race is punished. The team whose car scores the highest wins the contest\cite{rules}.

% Motivation, why is it interesting and important?

Even with such simple robots, there are enormous amounts of work, including the design of the system hardware carrying the sensor array, the software capable of recognizing and handling obstacles, a client software to ensure a safe testing environment and an efficient control system tasked to keep the the car stable while sweeping through the racetrack. As the fame of the competition grows, the expectations towards the cars heighten and teams become more ferocious to win. This all causes the pressure on the students to enlarge, while they have to hold their own in other challenges throughout the semester.

% Problem statement, why is it hard? (E.g., why do naive approaches fail?)?

Usually a low-level approach to the problem is the best approach to a small projects like this. Although good ideas people have used successfully in the past must not be forgotten, innovation is crucial. For these reasons, we decided to look beyond the boundaries of classic software development, to find a solution that enables us to concentrate on the important aspects, instead of being lost in the thousand lines of source code. However, utilizing a complex technology, and integrating several design softwares is no easy task. It might present more problems than solutions, because the available time for the development is very limited.

% Why hasn't it been solved before? (Or, what's wrong with previous proposed solutions? How does mine differ?)

Although these tools can greatly enhance the productivity of the developers, Model Based Design is still scarcely used in the industry, because of its initial setup complexity, and the high price of design softwares discourage its use even further. Against all these odds though its popularity is increasing, but it was never considered a way to deal with small projects before.

% What are the key components of my approach and results? Also include any specific limitations.

We wanted to demonstrate that a scaled-down approach is possible using this technology. Because the project is relatively small, the functional software and the operating system responsible for the core services can be integrated manually, and most of the advantages of the methodology can be retained while keeping the technology simple. We used \verb!MATLAB-Simulink! to create a \emph{Rapid Prototyping} environment, in which using a simulated model of the system conductive software development could be started weeks before an actual hardware prototype. Using automated code-generation for the target hardware, through a predefined interface the core services can be augmented with the proper functionalities, designed and verified in a swift visual model-based environment, to overcome the challenges the contest presents. The development cycle became shorter and the syntactic reliability of the resulting system is guaranteed by the code generation software.

\paragraph{Summary of Contributions}
\begin{itemize}
\item The paper will demonstrate how the focus of the software development has shifted from the actual coding to a more visual and mathematical representation, and how to harness its traits to quickly develop a reliable system.
\item It will be described how to use the \verb!MATLAB! \verb!Coder! auto code generation tool and how to integrate the source with the Core system, and deploy it on the target hardware, the \verb!STM32F4-Discovery! developer board.
\end{itemize}