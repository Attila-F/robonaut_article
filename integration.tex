\section{Integration}
\label{sec:integration}

\subsection{Structure of the code}

If during the code generation \emph{Compact code placement} was used, only four files are generated:

\begin{itemize}

    \item "system\_name".c
    \item "system\_name".h
    \item rtwtypes.h
    \item ert\_main.c

\end{itemize}

In the following we assume that the name of the Controller block was "controlsystem".

\begin{figure}[!ht]
    \centering
    \includegraphics[width=0.6\linewidth]{img/rtw}
    \caption{Relationship of the generated files}
    \label{fig:rtw}
\end{figure}

The functions of the system are defined in the \textbf{controlsystem.c} and \textbf{controlsystem.h} files. The \textbf{rtwtypes.h} contains the unique type definitions of the code. The \textbf{ert\_main.c} is an exemplary main function that demonstrates the use of the others.

\subsection{Using the code}

The generated source code is functionally a perfect equivalent of the \verb!Simulink! model. The system can be initialized using the \emph{controlsystem\_initialize()} function. It sets the 0 time in the model and the outputs obtain their initial values. Communication with the system is only possible through a unique interface provided by the generated code. The inputs are handled by the \emph{controlsystem\_U struct}. It contains fields each corresponding to a system input, matching it's name and type. If the inputs are set, calling the \emph{controlsystem\_step()} function runs the model once, for the period of one time sample. The outputs are stored in a structure similar to the input storage, called \emph{controlsystem\_Y}.

\subsection{Deployment with FreeRTOS on STM32F4-Discovery Board}

Freertos mint Core mit csinál? (beszélget a hardverrel, szenzorokkal, aktuárotoknak küld beavatkozó jeleket)
Miért jó? (gyors hozzáférés a szenzorokhoz, pontos ütemezés a controllernek)

\paragraph{Setting up the system}

Mit kellett csinálni, hogy működjön? (FreeRTOS-t feltenni rá (max 1 mondat, ebbe nem megyünk bele!), hardver jeleket kiolvasni C-vel, hogy oda tudjuk adni a controllernek, matlab-ot rákötni egy interrupt-ra)
Még mit kell csinálni, hogy működjön?? Max 1-2 mondat már csak.